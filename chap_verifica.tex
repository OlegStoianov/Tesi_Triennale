\chapter{Verifica e validazione}
\label{chap:verifica}

\section{Analisi statica}
Per l'analisi statica ed in particolare per alcune metriche come la complessità  ciclomatica, il numero di linee di codice per unità di compilazione e per le \gls{bestpractices} riguaranti la scrittura del codice è stato utilizzato SonarQube. 

\begin{figure}
\begin{center}
\includegraphics[width=0.5\columnwidth]{images/sonarqube-logo.png}
\end{center}
\caption{Logo SonarQube}
\label{fig:sonar}
\end{figure}

SonarQube è una piattaforma \gls{opensource} per la gestione della qualità  del codice. Nello specifico è un’applicazione web che produce \textit{reports} sul codice duplicato, sugli standards di programmazione, i tests di unità , il \textit{code coverage}, la complessità , i \textit{bugs} potenziali, i commenti, la progettazione e l’architettura \cite{sonarqube}.
\section{Analisi dinamica}
Per l'analisi dinamica non è stato utilizzato nessuno strumento in particolare, bensì durante le fasi di collaudo con il committente sono stati effettuati i vari test per verificare la correttezza del programma e dell'analisi effettuata. La fase finale di collaudo, effettuata le ultime due settimane prima del rilascio del progetto, è stata fondamentale per la correzione degli ultimi errori rimasti.
