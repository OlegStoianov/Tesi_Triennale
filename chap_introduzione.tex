\chapter{Introduzione}
\label{chap:introduzione}

\section{Scopo del documento}
\label{sec:scopodeldocumento}
Questo elaborato è la relazione finale sul lavoro svolto dal laureando Oleg Stoianov durante il periodo di stage presso l’azienda Innovery S.p.a come conclusione del percorso di laurea triennale. Lo scopo è descrivere dettagliatamente le mansioni svolte, dimostrando di aver compreso e messo in atto tutte le competenze acquisite durante questo corso di studi e complementate da una formazione specifica aziendale. \\



Il documento è strutturato in maniera tale da:
\begin{itemize}
    \item Dare un'idea del contesto lavorativo in cui è stato svolto lo stage
    \item Elencare gli obiettivi personali e aziendali
    \item Definire le richieste del committente del progetto
    \item Riportare la pianificazione del lavoro
    \item Riportare l'analisi effettuata
    \item Spiegare la soluzione proposta
    \item Descrivere le tecnologie e gli strumenti utilizzati
    \item Riportare il risultato, conclusioni e sviluppi futuri
\end{itemize}

\clearpage


\section{Inquadramento generale}
\label{sec:inqgenerale}
\hypertarget{introduzione}{}
Nel corso di tale attività ho avuto la possibilità di approfondire il mondo Web, in particolare ciò che riguarda le \textit{Web Application} e lo sviluppo di quest'ultime.

Le \textit{Web Application} sono delle applicazioni che implementano il  paradigma  Client/Server, ovvero un modello  di  interazione  tra  processi  software, che si suddividono tra \textit{client}, che richiedono i servizi, e \textit{server}, che offrono servizi. L’interazione  \textit{client}-\textit{server}  richiede  perciò  una  precisa  definizione  di  un’interfaccia  di  servizi,  che  elenca quelli messi a disposizione dal \textit{server}. Quindi il termine \textit{Web Application} indentifica un'applicazione risiedente  in  un  Server  Web  alla  quale  si  accede  tramite  Client Web, per esempio un \textit{browser},
o un altro programma con funzioni di navigazione operante secondo gli standard del \gls{WWW}.

Per quanto riguarda lo sviluppo della \textit{Web Application}, ho avuto modo di lavorare sia lato \textit{client} sia lato \textit{server}, occupando una mansione di sviluppatore \gls{FULL-STACK}. \\
Per lato \textit{server}, detto anche \textit{back-end} si intende tutta la parte logico-applicativa di un sito Web, con il quale l'utente interagisce indirettamente, che ne permette il funzionamento, la gestione e persistenza dei dati
e il reperimento degli stessi. Le tecnologie ed i \gls{framework} utilizzati in questo ambito sono state: Java Standard Edition, Java Enterprise Edition, Spring, Spring
Boot, Spring Data, Spring Security.\\
Per lato \textit{client}, detto anche \textit{front-end} si indica la parte visibile all’utente che naviga un sito web e con cui egli può interagire, tipicamente un’interfaccia utente. In questa parte è stato utilizzato principalmente il \gls{framework} Angular.\\
L'obiettivo della \textit{Web Application} è quello ottimizzare il processo di \textit{onborarding} dei flussi logici, ovvero permettere di velocizzare le operazioni di inserimento e di lettura dei dati all'interno di un database. I dati in questione, vengono poi letti da un'altra applicazione software, un \gls{systemintegrator}, principalmente utilizzato dall'azienda SIA S.p.A per la gestione, il monitoraggio, l'amministrazione, l'instradamento di volumi elevati di file in entrata e in uscita. Una volta letti, il \gls{systemintegrator} li utilizzerà per la creazione delle regole alla base dell'instradamento dei file. Nello specifco, all'arrivo di un file trasmesso da un \gls{businesspartner} A, il \gls{systemintegrator} controllerà le regole presenti all'interno del database ed in base a quest'ultime, incanalerà il file verso il \gls{businesspartner} B. 

\section{Convenzioni tipografiche}
Le convenzioni tipografiche utilizzate in questo documento sono:
\begin{itemize}
    \item Tutti i termini usati nel documento che richiedono una breve spiegazione aggiuntiva vengono riportati nel glossario, situato a fine documento.
    \item Le occorrenze dei termini riportati nel glossario sono riportate con \MYhref{glossario}{questa} \color{black}convenzione.
    \item I termini in lingua straniera o appartenenti al gergo tecnico sono evidenziati con il carattere \textit{corsivo}.
    
\end{itemize}






%This is a reference to a chapter \ref{chap:quo}. This is a reference to a figure \ref{fig:doge}. This is a reference to some code \ref{lst:hello}. This is a citation \cite{famous:paper}.

%\lstinputlisting[label=lst:hello, firstline=2, lastline=4, caption={I directly included a portion of a file}]{code/hello.py}

%\begin{lstlisting}[language=Java, label=lst:java, caption={Some code in another language than the default one}]
%public void prepare(AClass foo) {
%        AnotherClass bar = new AnotherClass(foo)
%}
%\end{lstlisting}

% DA RIMUOVERE - LOREM IPSUM PER DIMOSTRAZIONE
%\foreignlanguage{english}{\Blindtext}

%\begin{figure}
%\begin{center}
%\includegraphics[width=0.5\columnwidth]{images/doge.png}
%\end{center}
%\caption{This is not a figure. It's a caption.}
%\label{fig:doge}
%\end{figure}

%\section{}