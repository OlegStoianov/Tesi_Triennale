\chapter{Introduzione}
\label{chap:introduzione}

\section{Scopo del documento}
\label{sec:scopodeldocumento}
Questo documento è la relazione finale su quanto svolto durante il periodo di stage come conclusione del percorso di laurea triennale.   \\

Esso è strutturato in maniera tale da:
\begin{itemize}
    \item Dare un idea del contesto lavorativo in cui è stato svolto lo stage
    \item Definire le richieste del committente del progetto
    \item Riportare la pianificazione del lavoro
    \item Riportare l'analisi effettuata
    \item Spiegare la soluzione proposta
    \item Descrivere le tecnologie e gli strumenti utilizzati
    \item Riportare il risultato e le conclusioni
\end{itemize}


Il tutto utilizzando conoscenze pregresse, formazione specifica aziendale ed un affiancamento ad un Team di sviluppo.





%This is a reference to a chapter \ref{chap:quo}. This is a reference to a figure \ref{fig:doge}. This is a reference to some code \ref{lst:hello}. This is a citation \cite{famous:paper}.

%\lstinputlisting[label=lst:hello, firstline=2, lastline=4, caption={I directly included a portion of a file}]{code/hello.py}

%\begin{lstlisting}[language=Java, label=lst:java, caption={Some code in another language than the default one}]
%public void prepare(AClass foo) {
%        AnotherClass bar = new AnotherClass(foo)
%}
%\end{lstlisting}

% DA RIMUOVERE - LOREM IPSUM PER DIMOSTRAZIONE
%\foreignlanguage{english}{\Blindtext}

%\begin{figure}
%\begin{center}
%\includegraphics[width=0.5\columnwidth]{images/doge.png}
%\end{center}
%\caption{This is not a figure. It's a caption.}
%\label{fig:doge}
%\end{figure}

%\section{}