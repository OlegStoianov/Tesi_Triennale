\label{glossario}

\newglossaryentry{front-end}
{
    name=front-end,
    description={la parte visibile all'utente di un programma e con cui egli può interagire, tipicamente un'interfaccia utente}
}
\newglossaryentry{back-end}
{
    name=back-end,
    description={programma e architettura con il quale l'utente interagisce indirettamente, di solito attraverso l'utilizzo di un'applicazione front-end}
}
\newglossaryentry{FULL-STACK}
{
    name=FULL-STACK,
    description={identifica lo sviluppo di un intero sistema informatico o applicazione dal front-end al back-end e al codice software che li collega}
}

\newglossaryentry{bestpractices}
{
    name=best practices,
    description={ Modo di fare che garantisce i migliori risultati in specifiche e note circostanze}
}

\newglossaryentry{ict}{
    name=ICT,
    description={“Information and Communication Technologies”, ICT è l’insieme delle tecnologie che forniscono l’accesso alle informazioni attraverso le telecomunicazioni. A differenza dell’Information Technology, l’ICT è più focalizzata sulle tecnologie di comunicazione, come internet, reti wireless, telefoni cellulari e altri mezzi di comunicazione}
}

\newglossaryentry{applicazioneweb}
{
    name=Web Application,
    description={identifica un’applicazione  risiedente  in  un  Server  Web  alla  quale  si  accede  tramite  un  browser  Internet  o  un  altro  programma  con  funzioni  di  navigazione  operante secondo gli standard del World Wide Web}
}

\newglossaryentry{agile}
{
    name=Agile,
    description={Lo sviluppo agile consiste nel rilasciare rapidamente modifiche al software in piccole porzioni con l'obiettivo di migliorare la soddisfazione dei clienti. Con l'adozione dello sviluppo agile i vari team costituiti da pochi sviluppatori ciascuno collaborano direttamente con i rappresentanti aziendali tramite incontri periodici durante l'intero ciclo di vita dello sviluppo del software}
}


\newglossaryentry{systemintegrator}{
    name=System Integrator,
    description={Nel campo dell'IT gli integratori di sistemi connettono sistemi eterogenei in modo che questi possano comunicare, processare, salvare, categorizzare dati}
}

\newglossaryentry{B2B}{
    name=B2B,
    description={"business-to-business", descrive le transazioni commerciali che intercorrono tra imprese industriali, commerciali o di servizi all'interno dei cosiddetti mercati interorganizzativi}
}

\newglossaryentry{EDI}{
    name=EDI,
    description={"Electronic Data Interchange", è l'interscambio di dati tra sistemi informativi, attraverso un canale dedicato e in un formato definito in modo da non richiedere intervento umano salvo in casi eccezionali}
}

\newglossaryentry{WWW}{
    name=WWW,
    description={"World Wide Web", è uno dei principali servizi di Internet, che permette di navigare e usufruire di un insieme molto vasto di contenuti amatoriali e professionali (multimediali e non) collegati tra loro attraverso legami (link), e di ulteriori servizi accessibili a tutti o ad una parte selezionata degli utenti di Internet}
}

\newglossaryentry{framework}
{
    name=framework,
    description={astrazione che fornisce un insieme di classi, strumenti, utilità con funzionalità generiche, adattabili al domino applicativo specifico per l’utilizzo desiderato.
    Un framework fornisce una maniera standard di produrre applicazioni che seguono principi consolidati e ne facilita l’integrazione}
}

\newglossaryentry{businesspartner}{
    name=business partner,
    description={Soggetto che va ad affiancarsi ad un'impresa in un rapporto di collaborazione continuata, in merito alla distribuzione, promozione e vendita dei suoi prodotti}
}

\newglossaryentry{MFT}{
    name=MFT,
    description={"Managed File Transfer", Il trasferimento di file gestito si riferisce a un software o un servizio che gestisce il trasferimento sicuro di dati da un computer a un altro attraverso una rete (ad esempio Internet). Il software MFT è commercializzato per le aziende come alternativa all'utilizzo di soluzioni di trasferimento di file ad hoc, come FTP, HTTP e altri}
}

\newglossaryentry{BPML}{
    name=BPML,
    description={"Business Process Modeling Language", è un linguaggio basato su XML per la modellazione dei Business Process. Fornisce un modello di esecuzione astratto per processi aziendali collaborativi e transazionali basato sul concetto di una macchina a stati finiti transazionale}
}

\newglossaryentry{ip}{
    name=indirizzo IP,
    description={è un numero del datagramma IP che identifica univocamente un dispositivo detto host collegato a una rete informatica che utilizza l'Internet Protocol come protocollo di rete per l'instradamento/indirizzamento}
}

\newglossaryentry{tcpip}{
    name=TCP/IP,
    description={indica una famiglia di protocolli di rete legati da dipendenze d'uso su cui si basa il funzionamento logico della rete Internet, rappresenta lo standard  \textit{de facto} nell'ambito delle reti dati}
}

\newglossaryentry{charset}{
    name=charset,
    description={è l'associazione fra un insieme di codici numerici e i rispettivi caratteri di un determinato linguaggio}
}
\newglossaryentry{codiceabi}{
    name=codice ABI,
    description={"Associazione Bancaria Italiana", è un numero composto da cinque cifre e rappresenta l'istituto di credito. Ogni banca possiede un codice ABI che viene assegnato proprio dall'Associazione bancaria italiana}
}

\newglossaryentry{diagrammagantt}{
    name=diagramma di Gantt,
    description={usato principalmente nelle attività di project management, è costruito partendo da un asse orizzontale a rappresentazione dell'arco temporale totale del progetto e da un asse verticale a rappresentazione delle mansioni o attività che costituiscono il progetto}
}

\newglossaryentry{designpattern}{
    name=design pattern,
    description={è un concetto che può essere definito "una soluzione progettuale generale ad un problema ricorrente". Si tratta di una descrizione o modello logico da applicare per la risoluzione di un problema}
}

\newglossaryentry{API}{
    name=API,
    description={"Application Programming Interface", sono un set di definizioni e protocolli con i quali vengono realizzati e integrati software applicativi. Consentono la comunicazione tra prodotti o servizi con altri prodotti o servizi senza sapere come vengono implementati}
}

\newglossaryentry{DBMS}{
    name=DBMS,
    description={"Database Management System", è un sistema software progettato per consentire la creazione, la manipolazione e l'interrogazione efficiente di database}
}

\newglossaryentry{schema}{
    name=schema,
    description={In un dizionario di dati, uno schema di database indica come le entità che compongono il database si relazionano tra loro, comprese le tabelle, le viste, le procedure memorizzate e altro ancora}
}


\newglossaryentry{log}{
    name=log,
    description={file costituito da un elenco cronologico delle attività svolte da un sistema operativo, da un database o da altri programmi, generato per permettere una successiva verifica}
}

\newglossaryentry{UML}{
    name=UML,
    description={è un linguaggio che permette, tramite l’utilizzo di modelli visuali, di analizzare, descrivere, specificare e documentare un sistema software anche complesso}
}

\newglossaryentry{IDE}{
    name=IDE,
    description={"Integrated Development Environment", è un software che, in fase di programmazione, supporta i programmatori nello sviluppo del codice sorgente di un programma}
}

\newglossaryentry{repository}{
    name=repository,
    description={è un archivio in grado di contenere dati e relativi metadati. Può offrire un sistema di
versionamento in grado di tenere traccia delle modifiche effettuate al suo interno}
}

\newglossaryentry{debugging}{
    name=debugging,
    description={indica l'attività che consiste nell'individuazione e correzione da parte del programmatore di uno o più errori (bug) rilevati nel software}
}

\newglossaryentry{DAO}{
    name=DAO,
    description={"Data Access Object", è un pattern architetturale per la gestione della persistenza: si tratta fondamentalmente di una classe con relativi metodi che rappresenta un'entità tabellare di un DBMS relazionale}
}

\newglossaryentry{sistemidistribuiti}{
    name=sistemi distribuiti,
    description={è una porzione di software che assicura che un insieme di calcolatori appaiano come un unico sistema coerente agli utenti del sistema stesso}
}

\newglossaryentry{URL}{
    name=URL,
    description={"Uniform Resource Locator",  è una sequenza di caratteri che identifica univocamente l'indirizzo di una risorsa su una rete di computer, come ad esempio un documento, un'immagine, un video, tipicamente presente su un host server e resa accessibile a un client}
}

\newglossaryentry{HTTP}{
    name=HTTP,
    description={"HyperText Transfer Protocol", è un protocollo a livello applicativo usato come principale sistema per la trasmissione d'informazioni sul web}
}

\newglossaryentry{REST}{
    name=REST,
    description={"Representational State Transfer", rappresenta un sistema di trasmissione di dati su HTTP.Il principio fondamentale è l'esistenza di risorse (fonti di informazioni), a cui si può accedere tramite un identificatore globale}
}

\newglossaryentry{opensource}{
    name=open source,
    description={un software è reso tale per mezzo di una licenza attraverso cui i detentori dei diritti favoriscono la modifica, lo studio, l'utilizzo e la redistribuzione del codice sorgente. Caratteristica principale dunque delle licenze open source è la pubblicazione del codice sorgente}
}






 







