\chapter{Stage}
\label{chap:stage}


\section{Introduzione al progetto}
\label{sec:introstage}
Lo stage ha avuto una durata di 3 mesi, dal 07/10/2019 al 07/01/2020 presso la sede milanese di SIA S.p.A in Via Privata Francesco Gonin 36. \\

Come accennato nel Sommario, il contenuto dello stage verte principalmente sullo sviluppo di un applicazione WEB, commissionata dall'azienda SIA S.p.A all'azienda Innovery S.p.A (presso la quale io sono stato assunto in data 01/04/2019). Il progetto in questione ha come obiettivo principale l'ottimizzazione del processo di onboarding di Business Partners all'interno di un System Integrator, \textit{IBM Sterling B2B Integrator}, utilizzato per lo scambio di file \textit{B2B ed EDI}. \\

L'implementazione di questa applicazione si pone come traguardo il permettere al team \textit{Sistemisti File Transfer}, team che lavora all'interno dell'azienda SIA S.p.A. di ottimizzare il loro \textit{'day-to-day work'} nel implementare le varie regole per quanto riguarda il transito dei file attraverso le loro macchine. Perciò l'applicazione oltre ad essere integrata con l'implementazione architetturale presente, dovrà anche possedere una User Interface, facile ed intuibile che permetta al Team \textit{Sistemisti File Transfer} di essere veloci e flessibili con le richieste dei vari progetti. \\



\section{Obiettivi}
\label{sec:obiettivi}


\subsection{Obiettivi personali}
\label{subsec:obiettivipersonali}

Gli obiettivi che mi ero posto prima dell'inizio di questo progetto sono stati :

\begin{itemize}
    \item Approfondire le mie conoscenze nell'ambito sviluppo WEB, in particolar modo 
    dello sviluppo \textit{FULL-STACK}, vedendo perciò sia la parte riguardante al \textit{frontend} che la parte riguardante al \textit{backend}.
    
    \item Lavorare in un team di sviluppo con esperienza, che applichi le \textit{Best practices} per lo sviluppo e la metodologia \textit{Agile}.
    
    \item Acquisire competenze nello sviluppo di software che dovrà integrarsi con altre soluzioni già esistenti.
    
    \item Approfondire le mie conoscenze nell'ambito bancario/finanziario, per quanto riguarda pagamenti, carte, banche..
\end{itemize}

\subsection{Obiettivi dell'azienda}
\label{subsec:obiettiviazienda}
Gli obiettivi dell'azienda SIA S.p.A. per questo progetto sono stati:

\begin{itemize}
    \item Ottimizzare il processo di oboarding dei Business Partners nel System Integrator \textit{IBM Sterling B2B Integrator}.
    \item Aumentare il livello di customizzazione dei propri prodotti.
    \item Mantenere l'architettura già esistente intatta.
\end{itemize}



\section{Piano di lavoro}
\label{sec:pianodilavoro}

Per lo sviluppo si è utilizzata la metodologia \textit{Agile}, \textit{Scrum}. \\
\textit{Scrum} è un approccio basato sulla teoria del controllo empirico dei processi: le decisioni vengono prese sulla base dell’esperienza (empirismo).
Tutti gli aspetti del lavoro devono essere visibili ai responsabili del risultato finale (trasparenza). Per rendere trasparenti questi elementi, il Team Scrum ispeziona di frequente il prodotto mentre lo sta sviluppando (ispezione). Così il processo e il prodotto possono essere adattati immediatamente nel caso di nuove esigenze o di condizioni mutate del mercato (adattamento). \cite{scruminfo}




