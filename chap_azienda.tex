\chapter{Le aziende}
\label{chap:aziende}
\section{Innovery S.p.A.}
\label{sec:innovery}

\begin{figure}
\begin{center}
\includegraphics[width=0.5\columnwidth]{images/logo_innovery.png}
\end{center}
\caption{Logo Innovery S.p.A.}
\label{fig:logo_innovery}
\end{figure}

\subsection{Profilo aziendale}
\label{subsec:innoprofilo}


Innovery è una società multinazionale nata nel 2001, che opera nell’area dei servizi ICT per le medie e grandi Aziende. Negli anni ha esteso il suo portfolio a tutte le aree della sicurezza informatica, coprendone tutti gli aspetti. Questo ha permesso di ampliare il suo mercato spingendosi sui territori internazionali. Ad oggi, infatti conta 10 sedi in tutto il mondo, coprendo oltre il territorio italiano, anche quello spagnolo e latinoamericano. Innovery offre soluzioni e servizi personalizzati, per soddisfare le esigenze specifiche dei clienti, completi di progettazione, realizzazione e supporto.\cite{innovery}


\subsection{Servizi offerti}
\label{subsec:innoservizi}
I servizi sono articolati in 10 unità operative: Security, Managed Services, Big Data, Mobile Solutions, E-Business, Billing  System, High  Performance  Computing, Physical  Security, Health Care, Funded  Projects. \cite{innoveryservizi} \\


Tra i più rilevanti abbiamo : \begin{itemize}
    \item Security Privacy Governance \\ 
La Security Privacy Governance rappresenta lo sguardo più generale sulla sicurezza e la
privacy; è essenziale per una Privacy by Design \& by Default e per ottenere la conformità
alle norme sulla sicurezza e sulla privacy (GDPR).
    
    \item Identity \& Access Management \\ 
L’Identity \& Access Management (IAM) è il punto di partenza essenziale di ogni sistema
di gestione della sicurezza delle informazioni: si tratta di disporre di soluzioni integrate
che consentano l’identificazione di individui e componenti dei sistemi, e di stabilire
quando e quali azioni possano svolgere sulle diverse risorse aziendali.

    \item Ethical Hacking/Forensic Analysis \\
L’attività di Vulnerability Assessment e Penetration Testing (VA/PT) consente di
avere una piena consapevolezza dello stato dell’infrastruttura e di tracciare un’efficace
Remediation Roadmap delle vulnerabilità identificate

    \item Network Security \\
La messa in sicurezza delle reti è la prima tappa della roadmap verso un’infrastruttura tecnologica sicura. Per la sicurezza dei dati e delle infrastrutture offre servizi specialistici basati sulle migliori tecnologie: Firewall; Intrusion Detection \& Prevention System (IDS/IPS); End Point Protection System; Advanced Persistent Threat (APT) Defense System.

    \item B2B Integration \& Managed File Transfer \\
L’integrazione End-to-End e l’efficienza del flusso di transazioni fra i sistemi interni e la
propria “comunità di business” è garantito da robusti strumenti come l’IBM Sterling B2B Integrator and Managed File Transfer Solutions (Sterling MFT). Sulla base di tale piattaforma
tecnologica, Innovery offre soluzioni come l’Innovery User Gateway (IUG), una
console Web di gestione centralizzata delle funzionalità della piattaforma Sterling Integrator;
\end{itemize}

\clearpage

\section{SIA S.p.A.}
\label{sec:sia}

\begin{figure}
\begin{center}
\includegraphics[width=0.5\columnwidth]{images/logo_sia.png}
\end{center}
\caption{Logo SIA S.p.A.}
\label{fig:logo_sia}
\end{figure}

\subsection{Profilo aziendale}
\label{subsec:siaprofilo}
SIA - società controllata da CDP Equity - è leader europeo nella progettazione, realizzazione e gestione di infrastrutture e servizi tecnologici dedicati alle Istituzioni Finanziarie, Banche Centrali, Imprese e Pubbliche Amministrazioni, nei segmenti Card \& Merchant Solutions, Digital Payment Solutions e Capital Market \& Network Solutions. Il Gruppo SIA eroga servizi in oltre 50 paesi e opera anche attraverso controllate in Austria, Croazia, Germania, Grecia, Repubblica Ceca, Romania, Serbia, Slovacchia, Sudafrica e Ungheria. La società ha inoltre filiali in Belgio e Olanda e uffici di rappresentanza in Inghilterra e Polonia.\cite{sia}


\subsection{Servizi offerti}
\label{subsec:siaservizi}

Tra i servizi offerti: processing delle carte di credito e debito, pagamenti elettronici, servizi di rete, piattaforme per i mercati finanziari.\cite{siasoluzioni} \\

In particolar modo: 

\begin{itemize}
    \item Card Management \\
    Soluzione innovativa e sicura di card management per singoli istituti di credito, grandi gruppi bancari, processor di carte e aziende. SIA supporta gli issuer nell’emissione di tutti i tipi di carte.
    
    \item Sistemi di pagamento tra cui:
    \begin{itemize}
        \item Servizi di tesoreria 
        \item Processing
        \item Accesso al Clearing
        \item Settlement
        \item Grandi Basi Dati
    \end{itemize}
    
    \item Servizi di rete \\
    SIAnet Financial Ring, si tratta di un'infrastruttura di connettività multisede a 10 Gbps sicura, affidabile e a bassa latenza, accessibile tramite 9 PoP (Point of Presence) a Milano, Londra, New York, Budapest e Francoforte che offrono una soluzione di connettività diretta e dedicata ai principali CSD, broker e sedi di negoziazione, consentendo in tal modo di ridurre i costi legati alla complessità.
    
    
    \item Blockchain \\
    Gestione digitalizzata delle fideiussioni basata su tecnologia blockchain.
    
    \item Sicurezza fisica \\
    SIA intelliFENCE è il brand con cui Emmecom firma le sue soluzioni per l’antintrusione, la videosorveglianza e il safety: una suite modulare e integrata, basata sui protocolli di comunicazione standard del settore bancario e offerta nella logica più vicina alle esigenze del cliente.
\end{itemize}


















