\chapter{Le aziende}
\label{chap:aziende}
\section{Innovery S.p.A.}
\label{sec:innovery}

\begin{figure}
\begin{center}
\includegraphics[width=0.5\columnwidth]{images/logo_innovery.png}
\end{center}
\caption{Logo Innovery S.p.A.}
\label{fig:logo_innovery}
\end{figure}

\subsection{Profilo aziendale}
\label{subsec:innoprofilo}


Innovery è una società multinazionale nata nel 2001, che opera nell’area dei servizi \Gls{ict} per le medie e grandi Aziende. Negli anni ha esteso il suo portfolio a tutte le aree della sicurezza informatica, coprendone tutti gli aspetti. Questo ha permesso di ampliare il suo mercato spingendosi sui territori internazionali. Ad oggi, infatti conta 10 sedi in tutto il mondo, coprendo oltre il territorio italiano, anche quello spagnolo e latinoamericano. Innovery offre soluzioni e servizi personalizzati, per soddisfare le esigenze specifiche dei clienti, completi di progettazione, realizzazione e supporto \cite{innovery}.


\subsection{Servizi offerti}
\label{subsec:innoservizi}
I servizi sono articolati in 10 unità operative: \textit{Security, Managed Services, Big Data, Mobile Solutions, E-Business, Billing  System, High  Performance  Computing, Physical  Security, Health Care, Funded  Projects} \cite{innoveryservizi}.\\


Tra i più rilevanti abbiamo : \begin{itemize}
    \item \textit{Security Privacy Governance} \\ 
La \textit{Security Privacy Governance} rappresenta lo sguardo più generale sulla sicurezza e la
privacy; è essenziale per una \textit{Privacy by Design} \& \textit{by Default} e per ottenere la conformità
alle norme sulla sicurezza e sulla privacy (GDPR).

    \item \textit{Security Operation Center} \\
    La gestione unitaria della sicurezza dei sistemi richiede l’impiego di specifiche tecnologie e di team strategici di personale qualificato, in grado di intervenire su ogni aspetto dell’in‑frastruttura tecnologica e dell’organizzazione aziendale: si tratta dei \textit{Security Operation Center} (SOC). Una specifica business unit di Innovery fornisce soluzioni e servizi SOC che consentono il monitoraggio delle infrastrutture informatiche, il rilevamento di anomalie, la gestione degli incidenti. Tutti i servizi sono basati sulle più avanzate soluzioni di mercato per le attività di \textit{Security Information} e \textit{Event Management} ‑ SIEM (ArcSight, Splunk, RSA).
    
    \item \textit{Identity} \& \textit{Access Management} \\ 
\textit{L’Identity} \& \textit{Access Management} (IAM) è il punto di partenza essenziale di ogni sistema
di gestione della sicurezza delle informazioni: si tratta di disporre di soluzioni integrate
che consentano l’identificazione di individui e componenti dei sistemi, e di stabilire
quando e quali azioni possano svolgere sulle diverse risorse aziendali.

    \item \textit{Ethical Hacking/Forensic Analysis} \\
L’attività di \textit{Vulnerability Assessment e Penetration Testing} (VA/PT) consente di
avere una piena consapevolezza dello stato dell’infrastruttura e di tracciare un’efficace
\textit{Remediation Roadmap} delle vulnerabilità identificate

    \item \textit{Network Security} \\
La messa in sicurezza delle reti è la prima tappa della roadmap verso un’infrastruttura tecnologica sicura. Per la sicurezza dei dati e delle infrastrutture offre servizi specialistici basati sulle migliori tecnologie: \textit{Firewall}; \textit{Intrusion Detection \& Prevention System} (IDS/IPS); \textit{End Point Protection System}; \textit{Advanced Persistent Threat} (APT) \textit{Defense System}.

    \item \textit{B2B Integration} \& \textit{Managed File Transfer} \\
L’integrazione \textit{End-to-End} e l’efficienza del flusso di transazioni fra i sistemi interni e la
propria “comunità di business” è garantito da robusti strumenti come l’IBM Sterling B2B Integrator and \textit{Managed File Transfer Solutions} (Sterling MFT). Sulla base di tale piattaforma
tecnologica, Innovery offre soluzioni come l’Innovery User Gateway (IUG), una
console Web di gestione centralizzata delle funzionalità della piattaforma Sterling Integrator;
\end{itemize}

\clearpage

\section{SIA S.p.A.}
\label{sec:sia}

\begin{figure}
\begin{center}
\includegraphics[width=0.5\columnwidth]{images/logo_sia.png}
\end{center}
\caption{Logo SIA S.p.A.}
\label{fig:logo_sia}
\end{figure}

\subsection{Profilo aziendale}
\label{subsec:siaprofilo}
SIA - società controllata da CDP Equity - è leader europeo nella progettazione, realizzazione e gestione di infrastrutture e servizi tecnologici dedicati alle Istituzioni Finanziarie, Banche Centrali, Imprese e Pubbliche Amministrazioni, nei segmenti \textit{Card} \& \textit{Merchant Solutions, Digital Payment Solutions e Capital Market} \& \textit{Network Solutions}. Il Gruppo SIA eroga servizi in oltre 50 paesi \cite{sia}.


\subsection{Servizi offerti}
\label{subsec:siaservizi}

Tra i servizi offerti: processing delle carte di credito e debito, pagamenti elettronici, servizi di rete, piattaforme per i mercati finanziari \cite{siasoluzioni}. \\

In particolar modo: 

\begin{itemize}
    \item \textit{Card Management} \\
    Soluzione innovativa e sicura di card management per singoli istituti di credito, grandi gruppi bancari, processor di carte e aziende. SIA supporta gli issuer nell’emissione di tutti i tipi di carte.
    
    \item Sistemi di pagamento\\
    Lo sviluppo di strumenti di pagamento innovativi, in grado di sfruttare le potenzialità offerte dalla tecnologia informatica, consente di modernizzare le abitudini di pagamento dei cittadini, delle imprese e della Pubblica amministrazione, migliorare la fluidità delle transazioni, sostenere la crescita economica \cite{bancaitalia} .
    \\Le soluzioni si suddividono in:
    \begin{itemize}
        \item Servizi di tesoreria 
        \item \textit{Processing}
        \item Accesso al \textit{Clearing}
        \item \textit{Settlement}
    \end{itemize}
    
    \item Servizi di rete \\
    SIAnet Financial Ring, si tratta di un'infrastruttura di connettività multisede a 10 Gbps sicura, affidabile e a bassa latenza, accessibile tramite 9  \textit{Point of Presence} (PoP) a Milano, Londra, New York, Budapest e Francoforte che offrono una soluzione di connettività diretta e dedicata ai principali CSD, broker e sedi di negoziazione, consentendo in tal modo di ridurre i costi legati alla complessità.
    
    
    \item \textit{Blockchain} \\
    Gestione digitalizzata delle fideiussioni basata su tecnologia blockchain. Le garanzie fidejussorie sono per legge obbligatorie per alcune  tipologie di rapporti economici come, ad esempio, negli appalti pubblici nella P.A., nei contratti di fornitura tra aziende e nel versamento di anticipi per acquisto di case in costruzione tra privati. Alle garanzie fidejussorie sono legate alcune criticità come la falsificazione dei documenti, le frodi da parte di soggetti non abilitati, la gestione manuale delle pratiche in forma cartacea e i lunghi tempi del processo di rilascio. L'utilizzo della \textit{Distributed Ledger Technology} (DLT) permette di dematerializzare l’iter di rilascio delle polizze fideiussorie da parte di banche, intermediari finanziari, assicurazioni e certificare in modo univoco e irrevocabile tali garanzie.
    
    \item Sicurezza fisica \\
    SIA intelliFENCE è il brand con cui Emmecom firma le sue soluzioni per l’antintrusione, la videosorveglianza e il \textit{safety}: una suite modulare e integrata, basata sui protocolli di comunicazione standard del settore bancario e offerta nella logica più vicina alle esigenze del cliente.
    
    \item Gestione Documentale
    \begin{itemize}
        \item Fatturazione Elettronica \\
        Il processo di fatturazione elettronica, completamente automatizzato secondo una logica \textit{Straight Throught Processing} (STP), integrato con i sistemi informativi gestionali e con i sistemi di \textit{Corporate Banking}, consente di eliminare la discontinuità dei flussi informativi recuperare efficienza riconciliare in modo automatico le informazioni finanziarie e commerciali migliorare la gestione dei flussi di cassa.
        
        \item Conservazione Digitale \\
        Il servizio di conservazione digitale, conserva qualsiasi tipologia di documento e assicura gli appropriati criteri di ricerca. Il sistema supporta un set di tipologie documentali predefinite (contratti, ordinativi, fatture, log di sistema, etc) e permette di personalizzarle e crearne di nuove, tutte caratterizzate da dati di ricerca e dal \textit{workflow} di conservazione totalmente configurabil
    \end{itemize}
\end{itemize}


















